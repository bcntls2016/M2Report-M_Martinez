
In this work we describe the dynamics following photoexcitation of a potassium atom on a 4-helium nanodroplet.
4-Helium nanodroplets are ultra cold clusters (0.37~K) of typically $10^3$ up to $10^6$ helium atoms. 
This fascinating environment has been the focus of a whole community's interest for more than 30 years.
The reader should refer to the two nice review papers by Toennies et al. \cite{Toe2001, Toe2004} and the one by Barranco et al. \cite{Bar2006} which give a more detailed presentation of this topic.\\

At first, 4-helium droplets were studied for their own, intrinsic interest. 
Indeed, 4-helium is known to be superfluid at low temperature. 
Nevertheless superfluidity is usually defined as a macroscopic behavior. 
Hence these droplets are ideal candidates to address the definition of super-fluidity in a finite range system and its link to \textsc{Bose-Einstein} condensation. 
Several experimental evidences have been found to characterize their superfluidity.
The most intuitive one is the free rotation of molecules embedded in droplets: spectroscopy experiments exhibited resolved rotational spectra \cite{Har1995}, which is not possible within a classical fluid. 
However, the true evidence for superfluidity came from the excitation spectra of super-fluids predicted by \textsc{Landau} from hydrodynamic equations.
The existence of a roton gap and a critical velocity have been demonstrated in droplets \cite{Gre2000}. 
The quantum nature of droplets makes it possible to study other fundamental phenomena, such as quantized vortices \cite{Gom2014,Anc2017}. \\

Beyond their intrinsic fundamental interest, these have also become a true nano-laboratory thanks to their unique properties. 
We already talked about rotational spectroscopy study (most dopants are heliophilic and reside in the middle of the droplet). 
Droplets are also used to investigate alignment of molecules in an electromagnetic field within the droplet \cite{Mer2016,She2017}, the formationation of metal nanoclusters and nanowires \cite{Vol2016}, \textit{etc}. \cite{Toe2001, Toe2004}.\\

We are interested in the doping of a helium droplet by a single potassium atom. 
Alkali atoms have been demonstrated to reside in a dimple at the surface of the droplet \cite{Hig1998}, \textit{id est} they are one of the few species for which the potential interaction with He is weaker than the He-He interaction, which makes them heliophobic.\\

Most of alkali dopants have been intensively investigated by spectroscopy studies\cite{Sti1996,Lac2011,Lac2012,Bru2001,Her2012,Log2014,Log2015} which can probe accessible states for the attached impurity.
Upon excitation to non highly-repulsive states, the formation of exciplexes (stable Alk$^*$-He$_n$ with $n\sim 1-6$) or the desorption of free atom have been observed \cite{Lac2011,Bru2001,Reh2000A,Reh2000B,Sch2001}. 
These phenomena have been addressed from both the theoretical and the computational point of view within a $^4$He-time-dependent density functional theory ($^4$He-TDDFT) framework and a Diatomics In Molecules (DIM) model \cite{Log2015,Her2012,Van2017,Zbi2005}.
This method has proved to be the best compromise between accuracy and ability to describe droplets of realistic sizes.\\

We present in the following a $^4$He-TDDFT simulation of the dynamics following  of $(4p\gets 4s)$ and $(5s\gets 4s)$ excitation of potassium from the equilibrium configuration of a K-He$_N$ droplet with $N=1000$. 
The choice of potassium is motivated by a discrepancy in time resolved experimental studies \cite{Reh2000A,Reh2000B,Sch2001}. 
Furthermore, potassium is intermediate between the heavier alkalis (Rb, Cs) which could successfully be described using classical dynamics, and the lighter ones (Li, Na), which clearly require a quantum mechanical description.
In this work we test both treatment for the static, equilibrium properties and for the $(5s\gets 4s$) excitation.