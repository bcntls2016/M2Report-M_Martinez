The first goal of this study was to investigate the $4p \leftarrow 4s$ photo-excitation of a potassium doping a droplet in order to compare with experiments. 
We have shown that the qualitative behavior reported in both references \cite{Reh2000A,Reh2000B,Sch2001} was well described by our simulations.
In particularly, the formation of exciplexes was observed in $\Pi$ states and the ejection of a bare potassium atom in the $\Sigma$ state. 
We noted that the spatial resolution and the cubic grid were not optimal to describe the cylindrical symmetry underlying exciplexes formation.
This is why we only could make weak conclusions regarding the time scale, size and shape of exciplexes. 
Nevertheless, our results concerning their formation rate are clearly more consistent with \cite{Reh2000A,Reh2000B}.
This conclusion has been corroborated in a direct exchange with the authors of \cite{Sch2001}.
They told us that they could have failed in properly time-resolving the excitation dynamics because of the high repetition rate of their laser at the time. 
Finally, both references observed a dominant production of K$^*$He exciplexes and a significant proportion (10 \%) of K$^*$He$_2$ (and even larger exciplexes for \cite{Sch2001}), while our simulations tend to show only K-He exciplex (before symmetry breaking). 
The physical relevance of ring exciplex formation after symmetry breaking is not clear and need to be clarified.\\

The second point that is still under investigation at the moment, is the $5s \leftarrow 4s$ photo-excitation. 
There is no experimental literature on this transition, this study is then purely predictive. 
We have performed two different simulations, one with the potassium atom treated as a classical and one as a quantum particle.
 We observed that the potassium behavior is the same in both descriptions: it is ejected quickly from the droplet within 0.2 ps. 
 Nevertheless the asymptotic velocity is not the same.
This is why we focused our comparison on the study of an impulsive model giving an estimate of the number of closest atoms directly interacting with the potassium at the time of photoexcitation. 
We have shown that the quantum case gave a lower number than the classical one, which explains that there is less momentum given to K.\\

To conclude, this work demonstrates once again that He-TDDFT coupled with DIM model is a very powerful tool to address the dynamics of helium droplets doped by an alkali atom upon photo-excitation. 
We could give a theoretical contribution to choose between the two contradictory experimental results. 
However, we also underlined difficulty of the 4HeDFT-BCN-TLS code in describing exciplex formation due to a simulation grid unadapted to the cylindrical symmetry at small scale. 
Finally, we started to describe the potassium desorption upon excitation to the $5s$ state. 
We hope that our results will motivate experimentalists to work on both transitions.\\ 

In a close future, we would like to study the $4p \leftarrow 4s$ transition with test particles in order to evaluate the importance of quantum effects in this case.
We also want to address the question of the formation of a ring exciplex, and to finish our study of the $5s \leftarrow 4s$ transition. 
A long term motivation could be to rewrite part of the code to bypass the cubic limitation and get better results.

