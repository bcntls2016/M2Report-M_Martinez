\documentclass[a4paper,12 pt,twoside]{report}
\usepackage[utf8]{inputenc} 
\usepackage[T1]{fontenc}     
\usepackage[main = UKenglish,french]{babel}
\usepackage[top=2.4cm, bottom=2.4cm, left=1.4cm, right=1.4cm,headheight=14.5pt]{geometry}
%\usepackage[backend=bibtex]{biblatex}
%\bibliography{biblio.bib}

%	# Packages load

\usepackage[labelfont=bf,justification=centering]{caption}
% The cap­tion pack­age pro­vides many ways to cus­tomise the cap­tions in float­ing en­vi­ron­ments like fig­ure and ta­ble, and co­op­er­ates with many other pack­ages.
% url =  https://www.ctan.org/pkg/caption

\usepackage{amsmath}
% It adapts for use in LATEX most of the math­e­mat­i­cal fea­tures found in AMS-TEX; it is highly rec­om­mended as an ad­junct to se­ri­ous math­e­mat­i­cal type­set­ting in LATEX.
% url =  https://www.ctan.org/pkg/amsmath

\usepackage{amssymb}
% Provides an extended symbol collection.
% url =  ftp://ftp.dante.de/tex-archive/fonts/amsfonts/doc/amssymb.pdf

\usepackage{amsfonts}
% An ex­tended set of fonts for use in math­e­mat­ics.
% url =  https://www.ctan.org/pkg/amsfonts

\usepackage{enumitem}
% This pack­age pro­vides user con­trol over the lay­out of the three ba­sic list en­vi­ron­ments: enu­mer­ate, item­ize and de­scrip­tion.
% url =  https://www.ctan.org/pkg/enumitem

\usepackage{physics}
% The pack­age de­fines sim­ple and flex­i­ble macros for type­set­ting equa­tions in the lan­guages of vec­tor cal­cu­lus and lin­ear al­ge­bra, us­ing Dirac no­ta­tion.
% url =  https://www.ctan.org/pkg/physics

\usepackage{graphicx}
% The pack­age builds upon the graph­ics pack­age, pro­vid­ing a key-value in­ter­face for op­tional ar­gu­ments to the \in­clude­graph­ics com­mand.
% url =  https://www.ctan.org/pkg/graphicx

\usepackage{hyperref} 
% The hy­per­ref pack­age is used to han­dle cross-ref­er­enc­ing com­mands in LATEX to pro­duce hy­per­text links in the doc­u­ment.
% url =  https://www.ctan.org/pkg/hyperref
%\hypersetup{colorlinks=true, allcolors=blue}

\usepackage{fancyhdr} 
% The pack­age pro­vides ex­ten­sive fa­cil­i­ties, both for con­struct­ing head­ers and foot­ers, and for con­trol­ling their use
% url = https://www.ctan.org/pkg/fancyhdr

\usepackage{csquotes}
% This pack­age pro­vides ad­vanced fa­cil­i­ties for in­line and dis­play quo­ta­tions.
% url = https://www.ctan.org/pkg/csquotes

\usepackage{wrapfig}
% Al­lows fig­ures or ta­bles to have text wrapped around them.
% url =  https://www.ctan.org/pkg/wrapfig

\usepackage{subcaption}
% Permet de faire des sous-figures très facilement !

\usepackage{csquotes}
% Améliore les citations

\usepackage{indentfirst}
% Rajoute un alinéa tout le temps

% # Page style congiguration (fancyhdr)

\pagestyle{fancy}

\fancyhf{}% Clear header and footer
%\fancyhead[LE,RO]{\slshape  \rightmark}
%\fancyhead[LO,RE]{\leftmark}
\fancyfoot[C]{\thepage}% Custom footer
\renewcommand{\headrulewidth}{0pt}
\renewcommand{\footrulewidth}{0pt}

%\fancypagestyle{plain}
%{
%	\fancyhf{}%
%	\fancyfoot[C]{\thepage}%
%	\fancyhead[LE,RO]{\slshape  \rightmark}
%	\renewcommand{\headrulewidth}{0pt}
%}  

% # First page setup

\newcommand*{\titleGP}
{
	\begingroup % Create the command for including the title page in the document
	\newgeometry{top=1.4cm, bottom=0.8cm, left=1.4cm, right=1.4cm,headheight=14.5pt}
	\thispagestyle{empty}	
	\centering % Center all text
	\rule{\textwidth}{1.6pt}\vspace*{-\baselineskip}\vspace*{2pt} 
	\rule{\textwidth}{0.4pt}\\[\baselineskip] 
	\vspace*{-0.6\baselineskip}
	{\LARGE \textsc{Dynamics of a superfluid helium nanodroplet} \par} 
	\vspace*{0.2\baselineskip}
	{\LARGE \textsc{doped with a single potassium atom}}\\[0.15\baselineskip] % Title
	\rule{\textwidth}{0.4pt}\vspace*{-\baselineskip}\vspace{3.2pt} % Thin horizontal line
	\rule{\textwidth}{1.6pt}\\[\baselineskip] % Thick horizontal line
	\vfill
	\begin{figure}[h!]
		\centering
		\includegraphics[scale=0.35]{../pictures/couverture}
	\end{figure}	
	\vfill
	{\LARGE \textbf{Maxime \textsc{Martinez}$\,^{\circledast}$}\par} 
	\vspace*{0.1\baselineskip}
	{\large M2 \textit{Physique Fondamentale}, \textsc{Université Toulouse III Paul Sabatier}\par} 
	\vspace*{1.0\baselineskip}
	{\itshape Under the supervision of N.~\textsc{Halberstadt}$\,^{\circledast}$ and F.~\textsc{Coppens}$\,^{\circledast}$ \par}
	{\itshape In tight collaboration  with M.~\textsc{Barranco}$\,^{\circledast\circledcirc}$ and M.~\textsc{P\'i}$\,^{\circledcirc}$ \par}
	\vspace*{1.0\baselineskip}
	{\itshape \footnotesize $^{\circledast}$Laboratoire des Collisions, Agr\'egats, R\'eactivit\'e, IRSAMC, UMR 5589, CNRS et Universit\'e~Toulouse~III Paul~Sabatier, 118 route de Narbonne, Toulouse Cedex, France \par}
	\vspace*{0.1\baselineskip}
	{\itshape \footnotesize $^{\circledcirc}$Departament FQA, Facultat de F\'isica, and IN2UB, Universitat de Barcelona, Diagonal 645, 08028 Barcelona, Spain\par}
	\vspace*{-0.7\baselineskip}
	\begin{figure}[h!]
	\centering
	\begin{minipage}[c]{0.52\linewidth}
\centering
		\includegraphics[scale=0.7]{../pictures/logo-UPS}
	\end{minipage}
\hfill
	\begin{minipage}[c]{0.44\linewidth}
\centering
		\includegraphics[scale=0.9]{../pictures/logo-lcar}
	\end{minipage}
\end{figure}
	\restoregeometry
	\endgroup
}

\usepackage{tabularx}
\usepackage[explicit]{titlesec}

\usepackage{etoolbox}

\makeatletter
\patchcmd{\ttlh@hang}{\parindent\z@}{\parindent\z@\leavevmode}{}{}
\patchcmd{\ttlh@hang}{\noindent}{}{}{}
\makeatother
%\definecolor{darkblue}{rgb}{0.12,0.47,0.87}
\titleformat{\chapter}[display] 
    {\vspace{-4\baselineskip}\Huge \selectfont \bfseries}
    {\centering \thechapter.\,#1}
    {0pt}
    {\centering \LARGE}[\vspace{-1\baselineskip}]

\titleformat{name=\chapter,numberless}[display] 
    {\vspace{-4\baselineskip}\Huge \selectfont \bfseries}
    {\centering #1}
    {0pt}
    {\centering \LARGE}[\vspace{-1\baselineskip}]
    

\newcommand\he{$^4$He}
\newcommand\heN{$^4$He$_n$}
\newcommand\HE{\mathrm{He}}
\newcommand\K{\mathrm{K}}
\newcommand\DIM{\mathrm{ES}}
\newcommand\Sc{\mathcal{S}}
\newcommand\KHE{\mathrm{K-He}}
\newcommand\KHEN{{\mathrm{K-He}_N}}
\newcommand\SO{\mathrm{SO}}
\newcommand\LS{\mathrm{LS}}
\newcommand\LJ{\mathrm{LJ}}
\newcommand\J{\mathrm{J}}

\newcommand\num{\addtocounter{equation}{1}\tag{\theequation}} % numerotte dans align*
\newcommand\with{\quad \text{with} \quad}
\newcommand\e{\mathrm{e}}
\renewcommand{\arraystretch}{1.5} % aggrandi tableaux
\newcommand\cadre[1]{\vspace{0.5\baselineskip} % cadre
\fbox{
	\begin{minipage}{0.95\textwidth}
	#1
	\end{minipage}}
\vspace{0.5\baselineskip}}

% pour citer
\newcommand\citfig[1]{fig. \ref{#1}}
\newcommand\cittab[1]{table \ref{#1}}
\newcommand\citsec[1]{section \ref{#1}}
\newcommand\citeq[1]{eq. \ref{#1}}
\newcommand\citanx[1]{annex \ref{#1}}

